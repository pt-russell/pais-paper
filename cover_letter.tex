\documentclass{letter}
\usepackage{txfonts}

\setlength{\textheight}{10.2in} 
\setlength{\topmargin}{-0.9in}

\signature{Dr Simon Cotter (corresponding author), Dr Colin Cotter and Paul Russell}
\address{School of Mathematics \\ Alan Turing Building \\ Oxford
  Road\\Manchester\\ M13 9PL \\ UK \\\textit{email:} simon.cotter@manchester.ac.uk}

\begin{document}

\date{}
\begin{letter}{ }

\opening{Dear Prof. Higdon,}

I am writing with regards to our previous submission to the SIAM/ASA Journal of Uncertainty Quantification, entitled ``Parallel Adaptive Importance Sampling''. Please forgive us for the length of this cover letter, but we felt that we needed to address a few issues from our previous submission.

This manuscript had previously been submitted to the SIAM Journal for Scientific Computing, back in August 2015. We received 2 pretty positive reviews, although one review criticised our lack of a high dimensional example. The introduction at the time was misleading as application areas were mentioned (meteorology, oceanography, oil well diagnostic problems) where the unknowns in the inverse problem are high dimensional. One reviewer focussed on this, and as such we were asked not to resubmit unless we were able to demonstrate the performance of the algorithm in a very high dimensional setting, which of course is not feasible for this type of method. As such, we didn't feel that it would be worth our time in resubmitting to this particular journal.

As a result of this, we undertook a thorough rewrite of the paper, addressing as many of the other concerns of the referees as we could, emphasising on low-dimensional but non-Gaussian distributions, one of which arises from inverse problems for biochemical networks. We then considered which other journals would be a good fit for this work, and felt that SIAM UQ would be perfect. The paper was substantially different, and as such we didn't view it as a simple resubmission to another journal. Our mistake was in not highlighting this in the coverletter to the submission to SIAM UQ.

In the response letter, the associate editor stated that: \emph{"You will see that both referees reports are quite short -- that is my doing. I was aware (through the SIAM grapevine) that you had previously submitted this paper to SISC and that lengthy reviews had been obtained. So, in the interest of not using too much of people's time, I asked for indicative reviews."} This clearly indicates that this submission was not treated as a new submission, and the referees were instructed not to give a full review. Moreover, their reviews have in our view been unfairly coloured by the previous reviews from the SISC submission which were provided to them. In particular, they criticise us due to \emph{"the computed example provides no information about whether or not the proposed method will work in the (sizeable) applications mentioned in your introduction."} However, all mentions of such sizeable applications have been completely removed from the introduction. The referees were responding to the reviews of the original paper which were sent to them, rather than the new paper, where these issues are not present.

The changes to the paper since the initial submission have been comprehensive - almost all of the numerical experiments have been rerun, a new 2D example was added from the application of parameter determination for chemical networks. In particular we would highlight also that a new resampler has been developed in this second version, a greedy approximation of the full state-of-the-art resampler that was used in the original paper. The introduction was also completely rewritten.

We are however very keen to make adjustments to the paper in order to appease any legitimate concerns that the referees have highlighted. In particular, we have replaced the 2D problem at the end of the paper with a 6 dimensional multimodal posterior arising from inverse problems for mixture models. We have also completely rewritten the introduction for a second time, taking more care to reference more relevant literature from, among others, the adaptive importance sampling community. Hopefully the introduction now better reflects the content of the rest of the paper. Importantly, we have also added a short passage which directs interested/concerned readers to proofs of the consistency of ensemble adaptive importance sampling schemes.

We are keen to resubmit the manuscript to SIAM/ASA JUQ for a second time. In order to address the issues that have arisen, we have provided three copies of the paper. The first is a latexdiff pdf of the changes that were made for our submission to SIAM/ASA JUQ from the original submission to SISC. The second is a latexdiff pdf of the changes we have made in our latest submission. The third version is our new submission. We also provide detailed descriptions of how the changes address the concerns of the referees.

However, if you still believe that this manuscript and its contents are not suitable for publication in your journal, we would very much appreciate a swift decision, in order that we are able to submit it elsewhere.

We thank you very much for your time and patience in dealing with this manuscript. We look forward to hearing from you in due course.

\closing{Yours sincerely,}
\end{letter}

\end{document}
